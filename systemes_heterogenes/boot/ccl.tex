\section{Conclusion}
Pour terminer ce rapport, il est bon de préciser que dans chacun des points abordés (démarrage d'un ordinateur jusqu'au chargement du noyau d'un OS, démarrage d'un système Linux, changement de mot de passe root sous Linux si celui-ci a été oublié), il est possible que l'explication donnée ne soit pas exhaustive. 

Cela est dû à la multiplicité des possibilités en terme de matériel et de logiciel. De fait, tout couvrir en détail représenterait un travail bien plus titanesque que ce qui est possible dans le cadre de ce projet. 

Un exemple de ce qui n'a pas été traité est le couple UEFI/GPT. UEFI (Unified Extensible Firmware Interface) est un remplaçant à la technologie BIOS dont les spécifications ont été publiées en 2005 et qui tend à être prédominant sur les nouvelles cartes mères. GPT (GUID Partition Table, GUID signifiant Globally Unique IDentifiers) est un standard de tables de partitions qui vise à remplacer MBR et fait lui-même partie du standard UEFI. 

Ces deux technologies prennent une approche différente, mais comme d'autres, les traiter ici demanderait bien trop de temps. Il reste néanmoins que le présent rapport donne une vue d'ensemble de ce qui existe et apporte un degré de compréhension non-négligeable sur la procédure de démarrage d'un ordinateur Linux. 