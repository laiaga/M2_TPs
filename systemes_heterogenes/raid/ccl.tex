\section{Conclusion}

Si l'on résume : RAID est un concept trop  large pour être traité dans le détail. Il y a un grand nombre de déclinaisons qui peuvent encore être combinées entre elles pour profiter des avantages de chacune. Les RAID 0, 1 et 5 présentés ici sont les plus utilisés. 

L'idée générale qui lie ces différentes techniques entre elles est celle d'améliorer le stockage de données sur une grappe de disques durs, que ce soit en termes de résistance aux pannes, de sauvegarde de l'information ou encore de vitesse d'écriture/lecture. 

Cela implique souvent, mais pas systématiquement, la fragmentation des données à écrire entre différents disques : chaque disque récupère un bloc du fichier initial, ce qui permet de profiter du contrôleur de chaque disque en lecture comme en écriture et donc de multiplier la vitesse de ces opérations proportionnellement à la taille de la grappe. 