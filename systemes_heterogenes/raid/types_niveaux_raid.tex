\section{Niveaux de RAID}

\subsection{Niveaux standards}
Il y a 8 niveaux standards de RAID, numérotés de 0 à 7, mais les plus courants sont les 0, 1 et 5. 

\subsubsection{RAID 0 : \textit{striping}}
Cette technique vise à augmenter la vitesse d'écriture des données. Il s'agit de fragmenter les données à écrire entre chacun des disques de la grappe pour obtenir une vitesse d'écriture améliorée ; en effet, chaque disque ayant son propre contrôleur, les écritures peuvent se faire en parallèle. 

On parle d'écriture des données par bande (\textit{strips}) qui a donné son nom au niveau. Une bande correspond au bloc d'un index donné sur chacun des disques (cf \textsc{table ~\ref{striping}}).

\begin{table}[h]
    \centering
    \caption{\label{striping} Disques en RAID 0}
    \begin{tabular}{|c|c|c|c|}
        \hline
         & \textbf{Disque 1} & \textbf{Disque 2} & \textbf{Disque 3} \\
        \hline
        \textbf{Bande 1} & Bloc 1 & Bloc 2 & Bloc 3 \\
        \hline
        \textbf{Bande 1} & Bloc 4 & Bloc 5 & Bloc 6 \\
        \hline
        \textbf{Bande 1} & Bloc 7 & Bloc 8 & Bloc 9 \\
        \hline
    \end{tabular}
\end{table}

On voit que cette technique n'apporte rien en terme de qualité de service, car si l'un des disques tombe en panne tout le système est inaccessible. Par ailleurs elle implique des disques de taille identique car les bandes faisant la même taille sur chaque disque, c'est le disque le plus petit qui définit le nombre maximum de bandes (et donc la quantité de données) utilisable. 

\subsubsection{RAID 1 : \textit{mirroring}}
Ici l'objectif n'est plus la performance mais la gestion des pannes et la qualité de service. On écrit les données en parallèle sur chaque disque de la grappe pour en avoir une sauvegarde (cf \textsc{table ~\ref{striping}}). Le fait que l'écriture soit simultanée fait que les disques sont interchangeables à tout moment, et le système reste opérationnel tant qu'au moins un des disques de la grappe n'est pas KO. 

\begin{table}[h]
    \centering
    \caption{\label{striping} Disques en RAID 0}
    \begin{tabular}{|l|c|r|}
        \hline
        \textbf{Disque 1} & \textbf{Disque 2} & \textbf{Disque 3} \\
        \hline
        Bloc 1 & Bloc 1 & Bloc 1 \\
        \hline
        Bloc 2 & Bloc 2 & Bloc 2 \\
        \hline
        Bloc 3 & Bloc 3 & Bloc 3 \\
        \hline
    \end{tabular}
\end{table}

Le désavantage de cette méthode est son prix : comme la même donnée est inscrite sur chaque disque, on n'utilise effectivement, pour $n$ disques, que $\frac{1}{n}$ de la mémoire totale disponible ; il faut donc investir dans de grandes quantités de mémoire pour obtenir la même capacité de stockage qu'avec du RAID 0 par exemple.

\subsubsection{RAID 5 : \textit{disk array with block-interleaved distributed parity}}

\subsection{Niveaux combinés}