\section{Présentation générale de RAID}

RAID est un ensemble de techniques de répartition des données sur une grappe de disques durs, cela pouvant viser différents objectifs : quantité de mémoire augmentée à moindre coût, qualité de service améliorée, vitesse d'écriture des données plus élevée. 

Le terme est apparu à la fin des années 80 et se concentrait au départ sur l'obtention d'une grande quantité de mémoire en utilisant des disques durs peu onéreux à une époque ou le prix du mégaoctet de mémoire était encore bien plus élevé qu'aujourd'hui.

Aujourd'hui les techniques qui composent le RAID sont répartis un grand nombre de niveaux représentant chacun un concept différent, dont certains sont redondants les uns par rapport aux autres, obsolètes ou encore simplement peu utilisés. Nous allons donc passer en revue les plus utilisés.