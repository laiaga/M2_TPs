\documentclass[a4paper,11pt]{article}
\usepackage{exptech}
\usepackage{textcomp}
\usepackage{graphicx}
\usepackage{array}
\usepackage[babel=true]{csquotes}
\usepackage{url}
\usepackage{hyperref}
\usepackage{wrapfig}
\usepackage[export]{adjustbox}
\usepackage{titletoc}
\usepackage{array}


\title{
  \textbf{Systèmes hétérogènes}\\
  Technologies RAID
}
\markright{Systèmes hétérogènes - RAID}
\author{
\begin{minipage}{0.4\textwidth}
	\begin{flushleft} \large
		\emph{Auteurs :}\\
        Mohamed Amine \textsc{Nasseh}\\
		Alexandre \textsc{Leonardi}\\
	\end{flushleft}
\end{minipage}
\begin{minipage}{0.4\textwidth}
	\begin{flushright} \large
		\emph{Encadrant :} \\
		Roland \textsc{Agopian}\\
	\end{flushright}
\end{minipage}
}

\date{8 octobre 2016}

\begin{document}
\maketitle
\thispagestyle{empty}
\begin{abstract}
\textbf{RAID :} Présentation rapide de la technologie RAID (\textbf{R}edundant \textbf{A}rray of \textbf{I}ndependant \textbf{D}isks, bin que le I ait signifié \textbf{I}nexpensive par le passé), qui permet de mettre en réseau un grand nombre de périphériques de stockage et d'en tirer divers avantages. Notamment nous allons parler des :
\begin{itemize}
	\item Objectifs généraux de RAID
    \item Principales déclinaisons 
\end{itemize}
\end{abstract}
\pagebreak

\tableofcontents
\pagebreak


\section{Présentation générale de RAID}

RAID est un ensemble de techniques de répartition des données sur une grappe de disques durs, cela pouvant viser différents objectifs : quantité de mémoire augmentée à moindre coût, qualité de service améliorée, vitesse d'écriture des données plus élevée. 

Le terme est apparu à la fin des années 80 et se concentrait au départ sur l'obtention d'une grande quantité de mémoire en utilisant des disques durs peu onéreux à une époque ou le prix du mégaoctet de mémoire était encore bien plus élevé qu'aujourd'hui.

Aujourd'hui les techniques qui composent le RAID sont répartis un grand nombre de niveaux représentant chacun un concept différent, dont certains sont redondants les uns par rapport aux autres, obsolètes ou encore simplement peu utilisés. Nous allons donc passer en revue les plus utilisés.
\pagebreak
\section{Niveaux de RAID standards}
Il y a 8 niveaux standards de RAID, numérotés de 0 à 7, mais les plus courants sont les 0, 1 et 5. 

\subsection{RAID 0 : \textit{striping}}
Cette technique vise à augmenter la vitesse d'écriture des données. Il s'agit de fragmenter les données à écrire entre chacun des disques de la grappe pour obtenir une vitesse d'écriture améliorée ; en effet, chaque disque ayant son propre contrôleur, les écritures peuvent se faire en parallèle. 

On parle d'écriture des données par bande (\textit{strips}) qui a donné son nom au niveau. Une bande correspond au bloc d'un index donné sur chacun des disques (cf \textsc{table ~\ref{striping}}).

\begin{table}[h]
    \centering
    \caption{\label{striping} Disques en RAID 0}
    \begin{tabular}{|c|c|c|c|}
        \hline
         & \textbf{Disque 1} & \textbf{Disque 2} & \textbf{Disque 3} \\
        \hline
        \textbf{Bande 1} & Bloc 1 & Bloc 2 & Bloc 3 \\
        \hline
        \textbf{Bande 1} & Bloc 4 & Bloc 5 & Bloc 6 \\
        \hline
        \textbf{Bande 1} & Bloc 7 & Bloc 8 & Bloc 9 \\
        \hline
    \end{tabular}
\end{table}

On voit que cette technique n'apporte rien en terme de qualité de service, car si l'un des disques tombe en panne tout le système est inaccessible. Par ailleurs elle implique des disques de taille identique car les bandes faisant la même taille sur chaque disque, c'est le disque le plus petit qui définit le nombre maximum de bandes (et donc la quantité de données) utilisable. 

\subsection{RAID 1 : \textit{mirroring}}
Ici l'objectif n'est plus la performance mais la gestion des pannes et la qualité de service. On écrit les données en parallèle sur chaque disque de la grappe pour en avoir une sauvegarde (cf \textsc{table ~\ref{striping}}). Le fait que l'écriture soit simultanée fait que les disques sont interchangeables à tout moment, et le système reste opérationnel tant qu'au moins un des disques de la grappe n'est pas KO. 

\begin{table}[h]
    \centering
    \caption{\label{striping} Disques en RAID 0}
    \begin{tabular}{|l|c|r|}
        \hline
        \textbf{Disque 1} & \textbf{Disque 2} & \textbf{Disque 3} \\
        \hline
        Bloc 1 & Bloc 1 & Bloc 1 \\
        \hline
        Bloc 2 & Bloc 2 & Bloc 2 \\
        \hline
        Bloc 3 & Bloc 3 & Bloc 3 \\
        \hline
    \end{tabular}
\end{table}

Le désavantage de cette méthode est son prix : comme la même donnée est inscrite sur chaque disque, on n'utilise effectivement, pour $n$ disques, que $\frac{1}{n}$ de la mémoire totale disponible ; il faut donc investir dans de grandes quantités de mémoire pour obtenir la même capacité de stockage qu'avec du RAID 0 par exemple.

\subsection{RAID 5 : \textit{disk array with block-interleaved distributed parity}}
Il s'agit d'une évolution du RAID 4 qui est une évolution du RAID 3. L'idée est que, comme en RAID 0, les données vont être réparties entre les différents disques sous la forme de bandes. Cependant, un des disques, plutôt que de contenir un bloc de données, va contenir la parité des autres blocs de la bande, c'est-à-dire le XOR de l'ensemble des blocs de la bande (cf \textsc{table ~\ref{raid5}}). 

\begin{table}[h]
    \centering
    \caption{\label{raid5} Disques en RAID 5}
    \begin{tabular}{|c|c|c|c|}
        \hline
        \textbf{Disque 1} & \textbf{Disque 2} & \textbf{Disque 3} & \textbf{Disque 4}\\
        \hline
        Bloc 1 & Bloc 2 & Bloc 3 & Parité 1+2+3 \\
        \hline
        Bloc 4 & Partié 4+5+6 & Bloc 5 & Bloc 6 \\
        \hline
        Parité 7+8+9 & Bloc 7 & Bloc 8 & Bloc 9 \\
        \hline
    \end{tabular}
\end{table}

On obtient, en RAID 5, des avantages du RAID 0 :
\begin{itemize}
	\item vitesse d'écriture améliorée car on profite  du contrôleur de chaque disque de manière équivalente 
    \item optimisation de l'espace utilisé, pour une grappe $n$ disques, c'est $\frac{n-1}{n}$ de la mémoire qui est effectivement utilisée
\end{itemize}

On obtient également l'avantage du RAID 1 qui est une sécurité des données accrue, bien que dans une moindre mesure. En RAID 1 on pouvait se permettre de perdre $n-1$ disques tandis qu'en RAID 5 seul 1 peut tomber en panne sans conséquence : les blocs de données peuvent être recalculés grâce aux autres blocs de la bande et la parité ; les blocs de parité eux peuvent être recalculés de la même manière qu'ils ont été calculés lors de leur première écriture. 
\pagebreak
\section{Autres niveaux de RAID}
Il y a de nombreux autres niveaux de RAID, moins prépondérant que les 0, 1 et 5, à commencer par les niveaux 2 à 4.

 RAID 2 proposait une vérification des données écrites qui a été depuis lors intégrée aux contrôleurs des disques durs ; RAID 3 et 4 ont été raffinés en RAID 5, plus complexe mais plus efficace. 

RAID 6 et une version améliorée (défaillance possible de $n-1$ disques sur une grappe de $n$ disques) mais plus coûteuse à mettre en \oe{}uvre. 

Enfin il existe des combinaisons de niveaux de RAID, par exemple le RAID 01 par exemple répartit les données en bandes réparties sur $n$ disques (RAID 0) puis copie cette grappe de $n$ disques $k$ fois (RAID 1). 

Le RAID 15 est composé de $n$ sous-grappes de $k$ disques, chaque disque d'une sous-grappe contient les mêmes données (RAID 1) et les données sont réparties entre les sous-grappes selon le principe de RAID 5, $n-1$ contiennent les données et une contient la parité. 
\pagebreak
\section{Conclusion}
Pour terminer ce rapport, il est bon de préciser que dans chacun des points abordés (démarrage d'un ordinateur jusqu'au chargement du noyau d'un OS, démarrage d'un système Linux, changement de mot de passe root sous Linux si celui-ci a été oublié), il est possible que l'explication donnée ne soit pas exhaustive. 

Cela est dû à la multiplicité des possibilités en terme de matériel et de logiciel. De fait, tout couvrir en détail représenterait un travail bien plus titanesque que ce qui est possible dans le cadre de ce projet. 

Un exemple de ce qui n'a pas été traité est le couple UEFI/GPT. UEFI (Unified Extensible Firmware Interface) est un remplaçant à la technologie BIOS dont les spécifications ont été publiées en 2005 et qui tend à être prédominant sur les nouvelles cartes mères. GPT (GUID Partition Table, GUID signifiant Globally Unique IDentifiers) est un standard de tables de partitions qui vise à remplacer MBR et fait lui-même partie du standard UEFI. 

Ces deux technologies prennent une approche différente, mais comme d'autres, les traiter ici demanderait bien trop de temps. Il reste néanmoins que le présent rapport donne une vue d'ensemble de ce qui existe et apporte un degré de compréhension non-négligeable sur la procédure de démarrage d'un ordinateur Linux. 
\pagebreak
\nocite{*}
\bibliography{raid/biblio}

\end{document}